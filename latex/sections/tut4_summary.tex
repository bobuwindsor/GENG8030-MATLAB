%===================================
% KEY SUMMARY — Tutorial 4 Concepts
%===================================

\section*{Key Concepts and Common Pitfalls (Tutorial 4 Summary)}

\subsection*{1. Relational and Logical Operators}

MATLAB uses specific symbols for comparisons. A common source of bugs is confusing assignment with equality.

\begin{table}[h]
\centering
\begin{tabular}{|c|l|c|l|}
\hline
\textbf{Operator} & \textbf{Description} & \textbf{Operator} & \textbf{Description} \\
\hline
\texttt{==} & Equal to & \texttt{\textasciitilde=} & Not equal to \\
\texttt{<} & Less than & \texttt{<=} & Less than or equal to \\
\texttt{>} & Greater than & \texttt{>=} & Greater than or equal to \\
\hline
\end{tabular}
\end{table}

\textcolor{red}{\textbf{Pitfall:}}
Confusing \texttt{=} (assignment) with \texttt{==} (comparison).
\begin{lstlisting}
if x = 5   % Error! Assigns 5 to x inside the condition.
if x == 5  % Correct. Checks if x is equal to 5.
\end{lstlisting}

\subsubsection*{Logical Operators \& Short-Circuiting}
MATLAB distinguishes between element-wise and short-circuit operators.

\begin{itemize}
    \item \textbf{Element-wise (\texttt{\&}, \texttt{|}, \texttt{\textasciitilde}):} Operates on arrays. Returns an array of logicals.
    \item \textbf{Short-circuit (\texttt{\&\&}, \texttt{||}):} Operates on \textbf{scalars} only. Used primarily in \texttt{if} and \texttt{while} statements.
    \begin{itemize}
        \item \texttt{A \&\& B}: Evaluates A. If A is false, it stops (B is never evaluated).
        \item \texttt{A || B}: Evaluates A. If A is true, it stops (B is never evaluated).
    \end{itemize}
\end{itemize}

\textbf{Order of Precedence:}
\begin{enumerate}
    \item Arithmetic operations (\texttt{+}, \texttt{*}, \texttt{\^{}})
    \item Relational operations (\texttt{>}, \texttt{<}, \texttt{==})
    \item Logical operations (\texttt{\textasciitilde}, \texttt{\&}, \texttt{|})
\end{enumerate}

\subsection*{2. Conditional Branching}

\subsubsection*{The \texttt{if-elseif-else} Structure}
Evaluates expressions sequentially. The first true expression executes its block, and the structure terminates.

\begin{lstlisting}
if x < 0
    y = -x;
elseif x == 0
    y = 0;
else
    y = x^2;
end
\end{lstlisting}

\subsubsection*{The \texttt{switch} Structure}
An alternative to \texttt{if} when comparing a single variable against specific distinct values (cases). It is often more readable for discrete logic.

\begin{lstlisting}
switch units
    case {'inch', 'in'}
        y = x * 2.54;
    case {'meter', 'm'}
        y = x * 100;
    otherwise
        disp('Unknown unit');
end
\end{lstlisting}

\textcolor{red}{\textbf{Pitfall:}}
Using \texttt{switch} for range comparisons (e.g., $x < 5$). \texttt{switch} checks for \textbf{equality} only. Use \texttt{if} for ranges.

\subsection*{3. Iterative Structures (Loops)}

\subsubsection*{The \texttt{for} Loop}
Used when the number of iterations is known \textbf{before} the loop starts.
\begin{lstlisting}
for k = 1:2:10
    x(k) = k^2;
end
\end{lstlisting}
\textbf{Note:} If you iterate over a matrix \texttt{A} (\texttt{for k = A}), MATLAB iterates over the \textbf{columns} of A.

\subsubsection*{The \texttt{while} Loop}
Used when the number of iterations is unknown and depends on a condition (e.g., convergence errors).
\begin{lstlisting}
error = 100;
while error > 0.01
    % Update estimate
    % Update error
end
\end{lstlisting}

\textcolor{red}{\textbf{Pitfall:}}
Creating an \textbf{Infinite Loop}. You must ensure the variables inside the \texttt{while} condition change; otherwise, the loop never ends.
\begin{lstlisting}
x = 5;
while x > 0
    disp(x); 
    % Missing x = x - 1; -> Infinite loop!
end
\end{lstlisting}

\subsection*{4. Logical Indexing vs. The \texttt{find} Command}

Extracting data based on conditions is a core MATLAB skill.

\textbf{Method A: Logical Masking (Preferred for simple replacement)}
Returns a logical array (1s and 0s).
\begin{lstlisting}
A = [5, -2, 3];
mask = A < 0;    % mask = [0, 1, 0]
A(mask) = 0;     % A becomes [5, 0, 3]
\end{lstlisting}

\textbf{Method B: The \texttt{find} Command}
Returns the \textbf{indices} where the condition is true.
\begin{lstlisting}
indices = find(A < 0);  % indices = 2
\end{lstlisting}

\textcolor{red}{\textbf{Pitfall:}}
Using \texttt{find} when a logical mask suffices.
\begin{itemize}
    \item \textbf{Bad:} \texttt{A(find(A>5)) = 0;} (Slower, unnecessary function call)
    \item \textbf{Good:} \texttt{A(A>5) = 0;} (Faster, cleaner)
\end{itemize}

\subsection*{5. Performance: Pre-allocation}

MATLAB arrays are dynamic, but resizing them inside a loop is computationally expensive (slow). Always "pre-allocate" memory (reserve space) before the loop.

\textbf{Without Pre-allocation (Slow):}
\begin{lstlisting}
for k = 1:10000
    y(k) = k^2;  % MATLAB must resize 'y' 10,000 times!
end
\end{lstlisting}

\textbf{With Pre-allocation (Fast):}
\begin{lstlisting}
y = zeros(1, 10000); % Create full array first
for k = 1:10000
    y(k) = k^2;  % Fills existing slots
end
\end{lstlisting}

\subsection*{6. Loop Control: \texttt{break} vs \texttt{continue}}

\begin{itemize}
    \item \texttt{break}: Terminates the loop entirely. Execution jumps to the statement \textbf{after} the \texttt{end}.
    \item \texttt{continue}: Skips the rest of the \textbf{current iteration} and jumps to the next iteration.
\end{itemize}

\textbf{Example:}
\begin{lstlisting}
for k = 1:5
    if k == 2
        continue; % Skips 2, goes to 3
    end
    if k == 4
        break;    % Stops loop completely at 4
    end
    disp(k);      % Displays: 1, 3
end
\end{lstlisting}