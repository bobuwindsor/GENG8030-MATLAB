%===================================
% KEY SUMMARY — Tutorial 1 Concepts
%===================================

\section*{Key Concepts and Common Pitfalls (Tutorial 1 Summary)}

\subsection*{1. MATLAB Arithmetic and Precedence Rules}

MATLAB follows a strict order of precedence:
\begin{itemize}
    \item Parentheses
    \item Exponentiation
    \item Multiplication and division
    \item Addition and subtraction
\end{itemize}

Incorrect placement of parentheses can completely change results.
For example:
\[
27^{1/3} \neq 27^1/3
\]

\textcolor{red}{\textbf{Pitfall:}}
Students frequently misinterpret expressions such as:
\begin{lstlisting}
16^-1/2
16^(-1/2)
\end{lstlisting}
which produce different answers due to operator precedence.

\subsection*{2. Scalar Operations vs Mathematical Notation}

MATLAB syntax must be explicit:
\begin{itemize}
    \item Multiplication requires \texttt{*}
    \item Division requires clear parentheses
\end{itemize}

Example:
\begin{lstlisting}
(3*y)/(4*x-8)   % Correct
3*y/4*x-8       % Often misinterpreted
\end{lstlisting}

\textcolor{red}{\textbf{Pitfall:}} Missing parentheses leads to unintended evaluation order.

\subsection*{3. Numerical Limits: Overflow and Underflow}

MATLAB floating-point limits can produce:
\begin{itemize}
    \item \texttt{Inf} when numbers exceed \texttt{realmax}
    \item \texttt{0} or precision loss near \texttt{realmin}
\end{itemize}

Example concept:
\begin{lstlisting}
x1 = a*b*d;    % may overflow
x2 = a*(b*d);  % safer evaluation
\end{lstlisting}

\textcolor{red}{\textbf{Pitfall:}} Intermediate calculations may overflow even if final results are valid.

\subsection*{4. Built-in Functions and Units}

Key MATLAB functions:
\begin{itemize}
    \item \texttt{log()} = natural logarithm
    \item \texttt{log10()} = base-10 logarithm
    \item Trigonometric functions use radians
\end{itemize}

\textcolor{red}{\textbf{Pitfall:}}
Confusing \texttt{log()} with base-10 logarithm is a very common mistake.

\subsection*{5. Arrays and Vectorization}

MATLAB operates efficiently on arrays:
\begin{lstlisting}
u = 0:0.1:10;
w = 5*sin(u);
\end{lstlisting}

Vectorized operations compute many values at once.

\textcolor{red}{\textbf{Pitfall:}}
Using matrix operators instead of element-wise operators:
\begin{itemize}
    \item Use element-wise operators for arrays: \verb|.*|, \verb|./|, \verb|.^|.
\end{itemize}

\subsection*{6. Plotting Basics}

Core plotting workflow:
\begin{lstlisting}
plot(x,y)
xlabel('x')
ylabel('y')
grid on
\end{lstlisting}

Important steps:
\begin{itemize}
    \item Define domain first
    \item Use consistent units
    \item Label axes clearly
\end{itemize}

\textcolor{red}{\textbf{Pitfall:}}
Forgetting element-wise operators when computing functions for plotting.

\subsection*{7. Script Files and Execution Order}

When MATLAB executes a name:
\begin{enumerate}
    \item Checks variables
    \item Checks built-in commands
    \item Searches current folder
    \item Searches path
\end{enumerate}

\textcolor{red}{\textbf{Pitfall:}}
Naming scripts the same as MATLAB functions causes execution errors.

\subsection*{8. Engineering Problem-Solving Workflow}

Recommended steps:
\begin{itemize}
    \item Define inputs and outputs clearly
    \item Verify with simple hand calculations
    \item Perform a reality check on results
\end{itemize}

\textbf{Common Mistake:}
Trusting MATLAB output without verifying physical meaning or units.

\subsection*{9. Debugging Strategy}

Typical error types:
\begin{itemize}
    \item Syntax errors (missing brackets, commas)
    \item Runtime errors (division by zero)
\end{itemize}

Recommended debugging methods:
\begin{itemize}
    \item Remove semicolons to inspect values
    \item Test simplified cases
    \item Check intermediate variables
\end{itemize}
