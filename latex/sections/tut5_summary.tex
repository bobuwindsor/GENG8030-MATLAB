%===================================
% KEY SUMMARY — Tutorial 9 Concepts
%===================================

\section*{Key Concepts and Common Pitfalls (Tutorial 9 Summary)}

\subsection*{1. Numerical Integration (Quadrature)}

MATLAB provides two primary approaches for integration: using function handles (for mathematical formulas) or data points (for experimental data).

\subsubsection*{A. Integrating a Function Handle: \texttt{integral}}
Uses adaptive Simpson's rule. High accuracy.
\begin{itemize}
    \item \textbf{Syntax:} \texttt{q = integral(fun, a, b)}
    \item \textbf{Example:} $\int_{0}^{\pi} \sin(x) dx$
\end{itemize}
\begin{lstlisting}
fun = @(x) sin(x);
area = integral(fun, 0, pi); % Returns 2.0
\end{lstlisting}

\subsubsection*{B. Integrating Data Points: \texttt{trapz}}
Uses the Trapezoidal Rule. Used when you have vectors of data $x$ and $y$, not a formula.
\begin{itemize}
    \item \textbf{Syntax:} \texttt{area = trapz(x, y)}
\end{itemize}
\begin{lstlisting}
x = 0:0.1:pi;
y = sin(x);
area = trapz(x, y); % Approx 2.0 (depends on spacing)
\end{lstlisting}

\textcolor{red}{\textbf{Pitfall:}}
Confusing the two methods.

\begin{itemize}
    \item You cannot pass a vector to \texttt{integral}.
    \item You cannot pass a function handle to \texttt{trapz} (unless you evaluate it first).
\end{itemize}

\subsection*{2. Numerical Differentiation}

Differentiation is sensitive to "noise" in data. MATLAB uses the \texttt{diff} function to calculate differences between adjacent elements.

\textbf{Syntax:} \texttt{d = diff(x)}
\begin{itemize}
    \item Result vector is 1 element shorter than the input vector ($N-1$ elements).
    \item \textbf{Approximate Derivative:} $\frac{dy}{dx} \approx \frac{\Delta y}{\Delta x}$
\end{itemize}

\begin{lstlisting}
x = [0, 1, 2, 3];
y = x.^2;        % [0, 1, 4, 9]
dy = diff(y);    % [1, 3, 5] (Length is 3)
dx = diff(x);    % [1, 1, 1]
deriv = dy ./ dx; 
\end{lstlisting}

\textcolor{red}{\textbf{Pitfall:}}
Plotting the derivative against the original $x$ vector.
\begin{lstlisting}
plot(x, deriv) % Error! Vectors must be same length.
\end{lstlisting}
\textbf{Fix:} Use \texttt{x(1:end-1)} or calculate a midpoint vector for plotting.

\subsection*{3. Solving ODEs (\texttt{ode45})}

The workhorse for solving Ordinary Differential Equations in MATLAB is \texttt{ode45}. It solves systems of the form $\frac{dy}{dt} = f(t, y)$.

\subsubsection*{A. The Basic Syntax}
\texttt{[t, y] = ode45(ode\_fun, t\_span, initial\_conditions)}
\begin{itemize}
    \item \textbf{ode\_fun:} A handle \texttt{@(t, y) ...} that returns the column vector of derivatives.
    \item \textbf{t\_span:} \texttt{[t\_start, t\_end]}
    \item \textbf{initial\_conditions:} Vector of starting values for $y$ (and $y'$ if higher order).
\end{itemize}

\subsubsection*{B. Solving Higher-Order ODEs}
You must convert higher-order ODEs into a system of first-order ODEs using **State Variables**.


\textbf{Example:} Mass-Spring-Damper $\ddot{x} = \frac{1}{m}(F - c\dot{x} - kx)$
1. Let $x_1 = x$ (Position)
2. Let $x_2 = \dot{x}$ (Velocity)
3. Derivatives:
   \begin{itemize}
       \item $\dot{x}_1 = x_2$
       \item $\dot{x}_2 = \frac{1}{m}(F - c x_2 - k x_1)$
   \end{itemize}

\begin{lstlisting}
% Parameters: m=1, c=2, k=5, F=0
ode_sys = @(t, x) [x(2); (1/1)*(0 - 2*x(2) - 5*x(1))];
[t, sol] = ode45(ode_sys, [0, 10], [1; 0]); % Init: pos=1, vel=0
\end{lstlisting}

\textcolor{red}{\textbf{Pitfall:}}
The derivative function MUST return a \textbf{column vector}.
\begin{itemize}
    \item \textbf{Wrong:} \texttt{[x(2), -x(1)]} (Row vector)
    \item \textbf{Right:} \texttt{[x(2); -x(1)]} (Column vector)
\end{itemize}

\subsection*{4. ODE Events (Stopping Early)}

Sometimes you need to stop integration based on a condition (e.g., "stop when the rocket hits the ground, $h=0$"), not just time.

\textbf{Steps:}
1. Define an event function.
2. Set options using \texttt{odeset}.
3. Pass options to \texttt{ode45}.

\begin{lstlisting}
function [value, isterminal, direction] = my_event(t, y)
    value = y(1);     % Detect when y(1) (height) = 0
    isterminal = 1;   % 1 = Stop integration
    direction = -1;   % -1 = Only detect falling (neg slope)
end

% Usage
opts = odeset('Events', @my_event);
[t, y] = ode45(fun, [0, 100], [10; 0], opts);
\end{lstlisting}

\subsection*{5. Summary of Functions}

\noindent
\begin{center}
\begin{tabular}{|l|l|}
\hline
\textbf{Function} & \textbf{Purpose} \\
\hline
\texttt{integral(fun, a, b)} & Numerical integration of a formula \\
\texttt{trapz(x, y)} & Numerical integration of data arrays \\
\texttt{diff(x)} & Difference between adjacent elements \\
\texttt{gradient(M)} & Numerical gradient of a matrix \\
\texttt{ode45} & Standard ODE solver (Runge-Kutta) \\
\texttt{odeset} & Create options structure for ODE solvers \\
\hline
\end{tabular}
\end{center}

\subsection*{6. Control Systems: Transfer Functions and State Variable Form}

MATLAB's Control System Toolbox provides specialized tools for modeling and analyzing Linear Time-Invariant (LTI) systems. You can define these systems in two primary ways: Transfer Functions and State-Space models.

\subsubsection*{A. Transfer Functions (\texttt{tf})}
A transfer function represents the relationship between the output signal of a control system and the input signal, for all possible input values.

\textbf{Syntax:} \texttt{sys = tf(right, left)} 
\begin{itemize}
    \item \textbf{\texttt{right}}: A vector containing the coefficients on the right side of the equation, arranged in descending derivative order.
    \item \textbf{\texttt{left}}: A vector containing the coefficients on the left side of the equation, also arranged in descending derivative order.
\end{itemize}

\textbf{Example:}
Consider the differential equation: $5\ddot{y} + 7\dot{y} + 5y = 5\dot{f} + f(t)$.
\begin{lstlisting}
% Create the transfer function model form named sys
sys = tf([5, 1], [5, 7, 5]);

% Plot the unit step response for zero initial conditions
step(sys);
\end{lstlisting}

\subsubsection*{B. State-Space Form (\texttt{ss})}
For linear differential equations, you can organize your state variables into a standardized matrix format known as the state variable form. This is especially useful for systems with multiple interacting variables.

The standard state-space model relies on four matrices (A, B, C, and D):
\begin{align*}
    \dot{x} &= Ax + Bu \\
    y &= Cx + Du
\end{align*}

\textbf{Syntax:} \texttt{sys = ss(A, B, C, D)} 

\textbf{Specific Example: Mass-Spring-Damper System}

Let's model a mechanical system with mass $m = 2$, damping surface friction $c = 5$, and spring stiffness $k = 3$. The output we want to track is the position, so $y = x_1$.



[Image of a mass-spring-damper free body diagram]


\textbf{Step 1. The Starting Point: Newton's Second Law}

For a Mass-Spring-Damper system, the fundamental equation of motion is based on the sum of forces ($\Sigma F = ma$). The basic second-order differential equation is:
$$m\ddot{x} + c\dot{x} + kx = F$$
Where:
\begin{itemize}
    \item $m$ = mass
    \item $c$ = damping coefficient
    \item $k$ = spring stiffness
    \item $x$ = position
    \item $\dot{x}$ = velocity (first derivative of position)
    \item $\ddot{x}$ = acceleration (second derivative of position)
    \item $F$ = external force (often written as $u(t)$ in control systems)
\end{itemize}

If we rearrange this equation to solve for acceleration ($\ddot{x}$), we divide everything by $m$:
$$\ddot{x} = \frac{1}{m}(F - c\dot{x} - kx)$$

\textbf{Step 2. Defining the State Variables (Why $\dot{x}_1 = x_2$)}

\textcolor{blue}{\textit{To reduce this second-order equation into first-order equations}}, we invent new variables called "state variables" to represent the distinct physical states of the system (position and velocity).
\begin{itemize}
    \item Let $x_1$ = Position ($x$)
    \item Let $x_2$ = Velocity ($\dot{x}$)
\end{itemize}

Now, let's take the first derivative of $x_1$:
If $x_1$ is position, then taking its derivative with respect to time ($\dot{x}_1$) gives us velocity ($\dot{x}$). Since we already defined velocity as $x_2$, it mathematically follows that:
$$\dot{x}_1 = x_2$$

\textbf{Step 3. Substituting into the Original Equation (Why $\dot{x}_2 = \dots$)}

Now we need an equation for the derivative of our second state variable, $\dot{x}_2$. Since $x_2$ is velocity ($\dot{x}$), its derivative $\dot{x}_2$ is acceleration ($\ddot{x}$).

We go back to our rearranged equation of motion from Step 1:
$$\ddot{x} = \frac{1}{m}(F - c\dot{x} - kx)$$

Substitute our new state variables into this equation:
\begin{itemize}
    \item Replace acceleration ($\ddot{x}$) with $\dot{x}_2$.
    \item Replace velocity ($\dot{x}$) with $x_2$.
    \item Replace position ($x$) with $x_1$.
    \item Replace the input force $F$ with $u(t)$ (standard notation for inputs).
\end{itemize}

This gives us the final translated equation:
$$\dot{x}_2 = \frac{1}{m}u(t) - \frac{k}{m}x_1 - \frac{c}{m}x_2$$


\textbf{Mathematical Explanation of Matrices A, B, C, and D:}

These equations can be put into matrix form:
\begin{equation*}
    \begin{bmatrix} \dot{x}_1 \\ \dot{x}_2 \end{bmatrix} = \begin{bmatrix} 0 & 1 \\ -\frac{k}{m} & -\frac{c}{m} \end{bmatrix} \begin{bmatrix} x_1 \\ x_2 \end{bmatrix} + \begin{bmatrix} 0 \\ \frac{1}{m} \end{bmatrix} u(t)
\end{equation*}

\begin{itemize}
    \item \textbf{Matrix $A$ (System Matrix):} Defines the internal system dynamics based on the coefficients of $x_1$ and $x_2$. Here, $A = \begin{bmatrix} 0 & 1 \\ -k/m & -c/m \end{bmatrix}$.
    \item \textbf{Matrix $B$ (Input Matrix):} Defines how the external input $u(t)$ enters the system. Since the force only directly affects acceleration ($\dot{x}_2$), the first row is $0$ and the second is $1/m$, making $B = \begin{bmatrix} 0 \\ 1/m \end{bmatrix}$.
    \item \textbf{Matrix $C$ (Output Matrix):} Maps the state variables to the desired output. Since we want position ($y = x_1$), we take $1 \cdot x_1$ and $0 \cdot x_2$, so $C = \begin{bmatrix} 1 & 0 \end{bmatrix}$.
    \item \textbf{Matrix $D$ (Feedthrough Matrix):} Represents any direct routing from input to output. For most mechanical systems without direct feedthrough, $D = [0]$.
\end{itemize}

To build this in MATLAB using the \texttt{ss} function, we define the exact values for the A, B, C, and D matrices:
\begin{lstlisting}[language=MATLAB]
% Define parameters
m = 2; c = 5; k = 3;

% Define Matrices
A = [0, 1; -k/m, -c/m];
B = [0; 1/m];
C = [1, 0];
D = 0;

% Create the LTI state-space object
sys = ss(A, B, C, D);
\end{lstlisting}

\subsubsection*{C. Solving with \texttt{ode45} vs. State-Space}

While the State-Space (\texttt{ss}) form is incredibly powerful for Linear Time-Invariant (LTI) systems, MATLAB's \texttt{ode45} is a general-purpose numerical solver that can handle both linear and nonlinear systems. 

To use \texttt{ode45} for equations higher than order 1, you must also write the equation as a set of first-order equations (often called the Cauchy form or the state-variable form). However, instead of strictly passing defined matrices, you pass a function handle that computes the column vector of derivatives $f(t,y)$.

\textbf{Using \texttt{ode45} for the same Mass-Spring-Damper System:}
\begin{lstlisting}[language=MATLAB]
function xdot = msd(t, x)
    % Define parameters and a sample constant input force
    m = 2; c = 5; k = 3; u = 10; 
    
    % Define matrices for clean calculation
    A = [0, 1; -k/m, -c/m];
    B = [0; 1/m];
    
    % Return column vector of derivatives
    xdot = A*x + B*u; 
end

% Call ode45 in the main script
% For 0 <= t <= 5, with initial conditions x1(0)=0, x2(0)=0
[t, x] = ode45(@msd, [0, 5], [0; 0]);
\end{lstlisting}

\textbf{Key Differences for Engineering Applications:}
\begin{itemize}
    \item \textbf{Linearity Restrictions:} The \texttt{ss} command requires a strictly linear system. The \texttt{ode45} function can seamlessly simulate nonlinearities, such as a pendulum where the state equation relies on $-\frac{g}{L}\sin(x_1)$ instead of a linear coefficient.
    \item \textbf{Toolbox Integration:} Creating a \texttt{sys} object unlocks the Control System Toolbox, allowing you to use high-level, single-command analysis functions like \texttt{step(sys)}, \texttt{impulse(sys)}, and \texttt{lsim(sys)}. Conversely, \texttt{ode45} returns raw time and state data that you must plot and analyze manually using standard plotting commands like \texttt{plot(t, x(:,1))}.
\end{itemize}


\subsubsection*{C. Simulating Linear Time-Invariant (LTI) Objects}
Once your model is created (using either \texttt{tf} or \texttt{ss}), MATLAB has several functions to evaluate how it behaves over time.

\begin{itemize}
    \item \textbf{\texttt{step(sys)}}: Computes and plots the unit-step response of the LTI object sys.
    \item \textbf{\texttt{impulse(sys)}}: Computes and plots the unit-impulse response of the LTI object sys.
    \item \textbf{\texttt{initial(sys, x0)}}: Computes and plots the free response of the LTI object sys given in state-model form, for the initial conditions specified in the vector x0.
    \item \textbf{\texttt{lsim(sys, u, t)}}: Simulated time response. Computes and plots the response of the LTI object sys to the input specified by the vector u, at the times specified by the vector t.
\end{itemize}

\textbf{Creating Custom Inputs:} The \texttt{gensig} function makes it easy to construct periodic input functions.
\begin{itemize}
    \item \textbf{Syntax:} \texttt{[u, t] = gensig(type, period)} 
    \item The \texttt{type} can be defined as \texttt{'sin'}, \texttt{'square'}, or \texttt{'pulse'}.
\end{itemize}