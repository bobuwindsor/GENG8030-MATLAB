\documentclass[DIN, pagenumber=false, fontsize=11pt, parskip=half]{scrartcl}

\usepackage[english]{babel}
\usepackage[utf8]{inputenc}
\usepackage[T1]{fontenc}
\usepackage{textcomp}
\usepackage{amsmath} 
\usepackage{graphicx}

% for matlab code
% Remove 'bw' option to enable colored syntax highlighting
\usepackage[framed,numbered]{mcode}

\setlength{\parindent}{0em}

% Custom headers
\setkomafont{section}{\normalfont\bfseries\LARGE}
\setkomafont{subsection}{\normalfont\bfseries\Large}

\newcommand{\mytitle}[1]{{\noindent\Large\textbf{#1}}}
\newcommand{\mypart}[1]{\vspace{1cm}\hrule\vspace{0.5cm}\noindent\textbf{\LARGE #1}\vspace{0.5cm}\hrule\vspace{1cm}}
\newcommand{\myproblem}[1]{\subsection*{#1}}

%===================================
\begin{document}

\noindent\textbf{GENG8030-MATLAB} \hfill \textbf{University of Windsor}\\
Winter 2026 \hfill Bob Little\\

\mytitle{Course Notes \hfill \today}

%===================================
% TUTORIAL 1 SECTION
%===================================
\mypart{Tutorial 1: An Overview of MATLAB}

\myproblem{Problem 3}
Suppose that $x=5$ and $y=2$. Use MATLAB to compute the following, and check the results with a calculator.
\begin{enumerate}
    \item[a.] $(1-\frac{1}{x^{5}})^{-1}$
    \item[b.] $3\pi x^{2}$
    \item[c.] $\frac{3y}{4x-8}$
    \item[d.] $\frac{4(y-5)}{3x-6}$
\end{enumerate}

\begin{lstlisting}
clear; clc;
x = 5;
y = 2;

% a. (1 - 1/x^5)^-1
result_a = (1 - 1/x^5)^-1;

% b. 3 * pi * x^2
result_b = 3 * pi * x^2;

% c. (3*y) / (4*x - 8)
result_c = (3*y) / (4*x - 8);

% d. (4*(y - 5)) / (3*x - 6)
result_d = (4*(y - 5)) / (3*x - 6);

% Display results
disp(table(result_a, result_b, result_c, result_d));
\end{lstlisting}

%-----------------------------------
\myproblem{Problem 5}
Assuming that the variables a, b, c, d, and f are scalars, write MATLAB statements to compute and display the following expressions. Test your statements for the values $a=1.12$, $b=2.34$, $c=0.72$, $d=0.81$ and $f=19.83$.
\begin{itemize}
    \item $x=1+\frac{a}{b}+\frac{c}{f^{2}}$
    \item $r=\frac{1}{\frac{1}{a}+\frac{1}{b}+\frac{1}{c}+\frac{1}{d}}$
    \item $s=\frac{b-a}{d-c}$
    \item $y=ab\frac{1}{c}\frac{f^{2}}{2}$
\end{itemize}

\begin{lstlisting}
clear; clc;
a = 1.12; b = 2.34; c = 0.72; d = 0.81; f = 19.83;

x = 1 + a/b + c/f^2;
r = 1 / (1/a + 1/b + 1/c + 1/d);
s = (b - a) / (d - c);
y = a * b * (1/c) * (f^2/2);

disp(['x = ', num2str(x)]);
disp(['r = ', num2str(r)]);
disp(['s = ', num2str(s)]);
disp(['y = ', num2str(y)]);
\end{lstlisting}

%-----------------------------------
\myproblem{Problem 9}
The functions \texttt{realmax} and \texttt{realmin} give the largest and smallest possible numbers that can be handled by MATLAB.
Suppose you have variables $a=3\times10^{150}$, $b=5\times10^{200}$.
\begin{enumerate}
    \item[a.] Use MATLAB to calculate $c=ab$.
    \item[b.] Supposed $d=5\times10^{-200}$ use MATLAB to calculate $f=d/a$.
    \item[c.] Use MATLAB to calculate the product $x=abd$ two ways.
\end{enumerate}

\begin{lstlisting}
% Check limits
realmax
realmin

a = 3e150;
b = 5e200;

% a. Calculate c = a*b (Expect Overflow)
c = a * b 

% b. d = 5e-200, calculate f = d/a (Expect Underflow)
d = 5e-200;
f = d / a

% c. Calculate x = abd in two ways
x1 = a * b * d; % Risk of intermediate overflow
y = b * d;      
x2 = a * y;     % Safer calculation

disp(['Method 1: ', num2str(x1)]);
disp(['Method 2: ', num2str(x2)]);
\end{lstlisting}

%-----------------------------------
\myproblem{Problem 22}
Use MATLAB to calculate:
\begin{enumerate}
    \item[a.] $e^{(-2.1)^{3}}+3.47\log(14)+\sqrt[4]{287}$
    \item[b.] $(3.4)^{7}\log(14)+\sqrt[4]{287}$
    \item[c.] $\cos^{2}(\frac{4.12\pi}{6})$
    \item[d.] $\cos(\frac{4.12\pi}{6})^{2}$
\end{enumerate}

\begin{lstlisting}
% Note: Source likely implies log base 10 for "log(14)" in standard notation,
% but MATLAB's log() is natural log. Using log10() for base 10.
ans_a = exp((-2.1)^3) + 3.47 * log10(14) + nthroot(287, 4);
ans_b = (3.4)^7 * log10(14) + nthroot(287, 4);
ans_c = cos((4.12 * pi) / 6)^2;
ans_d = cos(((4.12 * pi) / 6)^2);
\end{lstlisting}

%-----------------------------------
\myproblem{Problem 27}
Use MATLAB to plot the function $T=7 \ln t-8e^{0.3t}$ over the interval $1\le t\le3$.

\begin{lstlisting}
t = 1:0.01:3; 
T = 7 .* log(t) - 8 .* exp(0.3 .* t);

plot(t, T);
title('Temperature vs Time');
xlabel('Time (min)');        
ylabel('Temperature (C)');   
grid on;
\end{lstlisting}

%-----------------------------------
\myproblem{Problem 30}
A cycloid is described by $x=r(\phi-\sin~\phi)$ and $y=r(1-\cos~\phi)$. Plot for $r=10$ and $0\le\phi\le4\pi$.

\begin{lstlisting}
r = 10;
phi = 0 : 0.01 : 4*pi; 
x = r .* (phi - sin(phi));
y = r .* (1 - cos(phi));

plot(x, y);
title('Cycloid Plot (r=10)');
xlabel('x'); ylabel('y');
axis equal; 
\end{lstlisting}

%-----------------------------------
\myproblem{Problem 34}
Develop a procedure for computing the length of side $c_{2}$ of the two-triangle figure given sides $b_{1}$, $b_{2}$, $c_{1}$ and angles $A_{1}$, $A_{2}$. Test with $b_{1}=200$, $b_{2}=1801$, $c_{1}=1201$, $A_{1}=120^{\circ}$, $A_{2}=100^{\circ}$.

\begin{lstlisting}
% Inputs 
b1 = 200; b2 = 1801; c1 = 1201; 
A1_deg = 120; A2_deg = 100;
A1 = deg2rad(A1_deg); A2 = deg2rad(A2_deg);

% 1. Find common side 'a' (Top Triangle Law of Cosines)
a_sq = b1^2 + c1^2 - 2*b1*c1*cos(A1);
a = sqrt(a_sq);

% 2. Find c2 (Bottom Triangle) solving quadratic: 
% c2^2 - (2*b2*cos(A2))*c2 + (b2^2 - a^2) = 0
coeff_A = 1;
coeff_B = -2 * b2 * cos(A2);
coeff_C = b2^2 - a_sq;

possible_c2 = roots([coeff_A, coeff_B, coeff_C]);
c2 = possible_c2(possible_c2 > 0); % Filter positive

disp(['Side c2: ', num2str(c2')]);
\end{lstlisting}

%-----------------------------------
\myproblem{Problem 35}
Write a script to compute the three roots of $x^{3}+ax^{2}+bx+c=0$.

\begin{lstlisting}
a = input('Enter a: ');
b = input('Enter b: ');
c = input('Enter c: ');
disp(roots([1, a, b, c]));
\end{lstlisting}


%===================================
% TUTORIAL 2 SECTION
%===================================
\newpage
\mypart{Tutorial 2: Numeric, Cell and Structure Arrays}

\myproblem{Problem 10}
Consider the array $A = \begin{bmatrix}1&4&2\\ 2&4&100\\ 7&9&7\\ 3&\pi&42\end{bmatrix}$ and $B = \ln(A)$.

Write MATLAB expressions to do the following:
\begin{enumerate}
    \item[a.] Select just the second row of B.
    \item[b.] Evaluate the sum of the second row of B.
    \item[c.] Multiply the second column of B and the first column of A element by element.
    \item[d.] Evaluate the maximum value in the vector resulting from element-by-element multiplication of the second column of B with the first column of A.
    \item[e.] Use element-by-element division to divide the first row of A by the first three elements of the third column of B. Evaluate the sum of the elements of the resulting vector.
\end{enumerate}

\begin{lstlisting}
% Define Matrix A
A = [1, 4, 2;
     2, 4, 100;
     7, 9, 7;
     3, pi, 42];

% Define Matrix B (Natural log is log() in MATLAB)
B = log(A);

% a. Select second row of B
part_a = B(2, :);

% b. Sum of second row of B
part_b = sum(B(2, :));

% c. Multiply 2nd col of B and 1st col of A element-wise
part_c = B(:, 2) .* A(:, 1);

% d. Max value of result from c
part_d = max(part_c);

% e. Divide 1st row of A by first 3 elements of 3rd col of B
% Note: A(1,:) is 1x3. B(1:3, 3) is 3x1. 
% We must transpose B's slice to match dimensions.
vec_e = A(1, :) ./ B(1:3, 3)'; 
part_e = sum(vec_e);

disp(['Sum (Part b): ', num2str(part_b)]);
disp(['Max (Part d): ', num2str(part_d)]);
disp(['Sum (Part e): ', num2str(part_e)]);
\end{lstlisting}

%-----------------------------------
\myproblem{Problem 11}
Create a three-dimensional array D whose three "layers" are matrices A, B, and C. Use MATLAB to find the largest element in each layer of D and the largest element in D.

\begin{lstlisting}
A = [3, -2, 1; 6, 8, -5; 7, 9, 10];
B = [6, 9, -4; 7, 5, 3; -8, 2, 1];
C = [-7, -5, 2; 10, 6, 1; 3, -9, 8];

% Create 3D array D
D(:, :, 1) = A;
D(:, :, 2) = B;
D(:, :, 3) = C;

% Largest element in each layer
max_layer_1 = max(max(D(:, :, 1)));
max_layer_2 = max(max(D(:, :, 2)));
max_layer_3 = max(max(D(:, :, 3)));

% Largest element in D
max_total = max(D(:));

disp(['Max Total: ', num2str(max_total)]);
\end{lstlisting}

%-----------------------------------
\myproblem{Problem 15}
Given matrices A, B, and C, verify the associative and commutative laws for addition.

\begin{lstlisting}
A = [-7, 11; 4, 9];
B = [4, -5; 12, -2];
C = [-3, -9; 7, 8];

% a. A + B + C
res_a = A + B + C;

% b. A - B + C
res_b = A - B + C;

% c. Verify Associative Law: (A+B)+C = A+(B+C)
check_assoc = isequal((A+B)+C, A+(B+C));

% d. Verify Commutative Law: A+B+C = B+C+A = A+C+B
term1 = A + B + C;
term2 = B + C + A;
term3 = A + C + B;
check_comm = isequal(term1, term2) && isequal(term2, term3);

if check_assoc && check_comm
    disp('Laws Verified');
else
    disp('Verification Failed');
end
\end{lstlisting}

%-----------------------------------
\myproblem{Problem 19}
Plot the function $f(x)=\frac{4 \cos x}{x+e^{-0.75x}}$ over the interval $-2\le x\le16$.

\begin{lstlisting}
x = -2 : 0.05 : 16; % Smooth interval
f = (4 .* cos(x)) ./ (x + exp(-0.75 .* x));

plot(x, f);
title('Plot of f(x)');
xlabel('x');
ylabel('f(x)');
grid on;
\end{lstlisting}

%-----------------------------------
\myproblem{Problem 22}
A ship travels on a straight line course described by $y=(200-5x)/6$. The ship starts when $x=-20$ and ends when $x=40$. Calculate the distance at closest approach to a lighthouse located at the origin (0,0) without using a plot.

\begin{lstlisting}
% Define path range
x = -20 : 0.01 : 40;
y = (200 - 5 .* x) ./ 6;

% Distance formula d = sqrt(x^2 + y^2)
distances = sqrt(x.^2 + y.^2);

% Find minimum distance
min_dist = min(distances);

disp(['Closest approach distance: ', num2str(min_dist), ' km']);
\end{lstlisting}

%-----------------------------------
\myproblem{Problem 23}
Calculate work done $W=FD$ for five segments of a path given force and distance data. Find (a) work for each segment and (b) total work.

\begin{lstlisting}
% Data vectors
Force = [400, 550, 700, 500, 600]; % Newtons
Distance = [3, 0.5, 0.75, 1.5, 5]; % Meters

% a. Work per segment (Element-wise multiplication)
Work_segments = Force .* Distance;

% b. Total work
Work_total = sum(Work_segments);

disp('Work per segment (J):');
disp(Work_segments);
disp(['Total Work (J): ', num2str(Work_total)]);
\end{lstlisting}

%-----------------------------------
\myproblem{Problem 27}
Calculate compression $x$ and potential energy $PE = \frac{1}{2}kx^2$ for five springs given Force $F=kx$ and spring constant $k$.

\begin{lstlisting}
% Data
F = [11, 7, 8, 10, 9];          % Force (N)
k = [1000, 600, 900, 1300, 700]; % Constant (N/m)

% a. Compression x = F / k
x = F ./ k;

% b. Potential Energy PE = 0.5 * k * x^2
PE = 0.5 .* k .* (x.^2);

% Display results table
disp(table(F', k', x', PE', 'VariableNames', {'Force','k','Compression','PE'}));
\end{lstlisting}

%-----------------------------------
\myproblem{Problem 41}
Solve the following system using the left-division method.
\[
\begin{aligned}
6x - 3y + 4z &= 41 \\
12x + 5y - 7z &= -26 \\
-5x + 2y - 6z &= 16
\end{aligned}
\]

\begin{lstlisting}
% Coefficient Matrix A
A = [ 6, -3,  4;
     12,  5, -7;
     -5,  2, -6];

% Constant Vector B
B = [41; -26; 16];

% Solve for X = [x; y; z] using left division
Solution = A \ B;

disp('Solution [x; y; z]:');
disp(Solution);
\end{lstlisting}

\end{document}